\documentclass[../../why-computer.tex]{subfiles}
\begin{document}
    So, we discussed briefly what information is, short history of computers and what hardware is. But how does it all interact with each other to give us experience
    we used to call normal computer work? \par

    To put it simply --- hardware is the bare bones, blood system, muscles and joints of the computer. However, it's just laying there, doing nothing, \emph{until we tell it 
    what to do}. Where does the impulse come from? How can we step up from all the components just assembled together to interactions between them, producing some
    useful work for us? \emph{Well, we must program it}. \par

    Let's say, that you want to view some photos, saved on your disk in your laptop. What actions are you gonna do to view those photos? I imagine something similar to
    this: 

    \begin{enumerate}
        \item Turn the power on
        \item Wait for display to show something
        \item See welcome screen, once your OS finished booting up
        \item Enter your password to log in
        \item Locate your photos somewhere on computer
        \item Click on it, for application to appear and show your image
    \end{enumerate}

    \emph{Everything in this list, except for turning the power on relied on some software to work correctly}. It is software, that displaying things on your screen.
    It is software searches for OS files to boot up computer, it is software transforming your keyboard typing to something computer can understand, it is software
    waits for the correct password to be entered, it is software shows you neat folders and files for you to navigate, it is software starting application to show you
    photos. There is a \emph{ton of software involved} in such a simple process, isn't it? \par

    We also must answer for a crucial question: `what is software?'. Well, it's instructions, telling the computer what to do\footnote{\href{https://www.britannica.com/technology/software}
    {https://www.britannica.com/technology/software}}. Software is just a fancy term for computer programs. \par

    So, how does one tell computer what to do? Well, mostly, it boils down to tell \emph{processor} what to do\footnote{I purposefully don't write `CPU' instead of processor 
    here since some programs run on GPU processors}. As was discussed in `\nameref{section:modern-computers}' on page~\pageref{section:modern-computers}, processors are able 
    to receive and implement instructions in \emph{machine code}. Machine code is always represented in binary format, meaning it's just `raw' zeros and ones. It's very
    inconvenient for human to send something in binary format --- too cumbersome and error-prone. 
    
\end{document}