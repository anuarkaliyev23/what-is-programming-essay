\documentclass[../../why-computer.tex]{subfiles}
\begin{document}
	\subsection{What is software?}\label{subsection:what-is-software}
        So, we discussed briefly what information is, short history of computers and what hardware is. But how does it all interact with each other to give us experience
        we used to call normal computer work? \par

        To put it simply --- hardware is the bare bones, blood system, muscles and joints of the computer. However, it's just laying there, doing nothing, \emph{until we tell it 
        what to do}. Where does the impulse come from? How can we step up from all the components just assembled together to interactions between them, producing some
        useful work for us? \emph{Well, we must program it}. \par

        Let's say, that you want to view some photos, saved on your disk in your laptop. What actions are you gonna do to view those photos? I imagine something similar to
        this: 

        \begin{enumerate}
        \item Turn the power on
        \item Wait for display to show something
        \item See welcome screen, once your OS finished booting up
        \item Enter your password to log in
        \item Locate your photos somewhere on computer
        \item Click on it, for application to appear and show your image
        \end{enumerate}

        \emph{Everything in this list, except for turning the power on relied on some software to work correctly}. It is software, that displaying things on your screen.
        It is software searches for OS files to boot up computer, it is software transforming your keyboard typing to something computer can understand, it is software
        waits for the correct password to be entered, it is software shows you neat folders and files for you to navigate, it is software starting application to show you
        photos. There is a \emph{ton of software involved} in such a simple process, isn't it? \par



        We also must answer for a crucial question: `what is software?'. Well, it's instructions, telling the computer what to do\footnote{\href{https://www.britannica.com/technology/software}
        {https://www.britannica.com/technology/software}}. Software is just a fancy term for computer programs. \par

        \subsection{Behold! An assembly language}\label{subsection:behold-assembly-language}

        So, how does one tell computer what to do? Well, mostly, it boils down to tell \emph{processor} what to do\footnote{I purposefully don't write `CPU' instead of processor 
        here since some programs run on GPU processors}. As was discussed in `\nameref{section:modern-computers}' on page~\pageref{section:modern-computers}, processors are able 
        to receive and implement instructions in \emph{machine code}. Machine code is always represented in binary format, meaning it's just `raw' zeros and ones. It's very
        inconvenient for human to send something in binary format --- too cumbersome and error-prone. \par

        To overcome this nuisance, we must create a `translator' of sorts --- for us to write in some way, comfortable enough for human and to translate it into said machine code
        which is so convenient for the machines. \emph{Thus, we firstly created a distinction with something conventient for humans and for the machines}. Anyhow, one of the first
        such \emph{abstractions} over machine code is an \emph{assembly language}. Main purpose of assembly language is to translate \emph{as directly as possible} to machine code 
        instructions of the processor. Instead of being raw binary --- it is actually a \emph{language}, in this context, meaning that it can be written as text. Just to illustrate
        what it may look like, we can look at MS-DOS sources\footnote{\href{https://github.com/microsoft/MS-DOS}{https://github.com/microsoft/MS-DOS}}. Code presented above can be found in file `PRINT.ASM'\footnote{\href{https://github.com/microsoft/MS-DOS/blob/master/v2.0/source/PRINT.ASM}
        {https://github.com/microsoft/MS-DOS/blob/master/v2.0/source/PRINT.ASM}} in MS-DOS repository. \par

    \begin{verbatim}
;Interrupt routines
ASSUME  CS:DG,DS:NOTHING,ES:NOTHING,SS:NOTHING
        IF      HARDINT
HDSPINT:                        ;Hardware interrupt entry point
        INC     [TICKCNT]       ;Tick
        INC     [TICKSUB]       ;Tick
        CMP     [SLICECNT],0
        JZ      TIMENOW
        DEC     [SLICECNT]      ;Count down
        JMP     SHORT CHAININT  ;Not time yet
TIMENOW:
        CMP     [BUSY],0        ;See if interrupting ourself
        JNZ     CHAININT
        PUSH    DS
        PUSH    SI
        LDS     SI,[INDOS]      ;Check for making DOS calls
        CMP     BYTE PTR [SI],0
        POP     SI
        POP     DS
        JNZ     CHAININT        ;DOS is Buisy
        INC     [BUSY]          ;Exclude furthur interrupts
        MOV     [TICKCNT],0     ;Reset tick counter
        MOV     [TICKSUB],0     ;Reset tick counter
        STI                     ;Keep things rolling

        IF      AINT
        MOV     AL,EOI          ;Acknowledge interrupt
        OUT     AKPORT,AL
        ENDIF

        CALL    DOINT
        CLI
        MOV     [SLICECNT],TIMESLICE    ;Either soft or hard int resets time slice
        MOV     [BUSY],0        ;Done, let others in
CHAININT:
        JMP     [NEXTINT]       ;Chain to next clock routine
        ENDIF


SPINT:                          ;INT 28H entry point
        IF      HARDINT
        CMP     [BUSY],0
        JNZ     NXTSP
        INC     [BUSY]          ;Exclude hardware interrupt
        INC     [SOFINT]        ;Indicate a software int in progress
        ENDIF

        STI                     ;Hardware interrupts ok on INT 28H entry
        CALL    DOINT

        IF      HARDINT
        CLI
        MOV     [SOFINT],0      ;Indicate INT done
        MOV     [SLICECNT],TIMESLICE    ;Either soft or hard int resets time slice
        MOV     [BUSY],0
        ENDIF

NXTSP:  JMP     [SPNEXT]        ;Chain to next INT 28

DOINT:
        PUSH    SI
        MOV     SI,[CURRFIL]
        INC     SI
        INC     SI
        CMP     BYTE PTR CS:[SI],-1
        POP     SI
        JNZ     GOAHEAD
        JMP     SPRET           ;Nothing to do
GOAHEAD:
        PUSH    AX              ;Need a working register
        MOV     [SSsave],SS
        MOV     [SPsave],SP
        MOV     AX,CS
        CLI
;Go to internal stack to prevent INT 24 overflowing system stack
        MOV     SS,AX
        MOV     SP,OFFSET DG:ISTACK
        STI
        PUSH    ES
        PUSH    DS
        PUSH    BX
        PUSH    CX
        PUSH    DX
        PUSH    SI
        PUSH    DI
        PUSH    CS
        POP     DS
    \end{verbatim}
    
        Don't worry if you don't understand a thing --- me neither\footnote{My apologies for assembly language teachers in university}. Main idea of this demonstration is to
        show that `convenient for humans' is a bit of a stretch regarding design of the assembly language. It's just not the case for the overwhelming part of programmers 
        worldwide, who can decipher those instructions \emph{a little} more than total laymen. How come? \par

        Well, although to write in assembly language --- is the next level of coolness in my eyes, it just isn't necessary most of the times. Some time ago, indeed, all the
        programmers had to write in some sort of assembly language to program anything, but those times \emph{mostly} far from now. Why? Well, have you seen an assembly language? It is 
        scary first of all! But \emph{it is a matter of habit} --- anyone could learn it and get used to it. Hey, if it get things done, why not? But there is at least one
        far more fundamental issue with assembly language: did you notice how I said `programmers had to write in \emph{some sort} of assembly language' a couple of sentences 
        ago? It's not a figure of speech --- \emph{there are many kinds of assembly language}. \par

        Since assembly \emph{languages} designed to be as close to the underlying machine code as possible, it's design depends heavily on said machine code. Well, who defines
        machine code? \emph{Processor's vendor does}. Processors vary greatly in many ways and \emph{there is just no standard, that can unionize them all under 
        some standardised machine code}. In a practical sense, it means that everything you write and test on your computer may just not be able to work on some other computer
        if it uses different processor. Well, as we discussed in section `\nameref{section:simplification-through-standartisation}', no standartisaion = no scalability. \par
        
        \begin{quote}
                Despite all of the issues discussed here, programmers still use `pure' assembly language sometimes. It is absolutely crucial if you work on something
                that must use some hardware-specific features or you just must have total control over some hardware-specific behaviour.
                It is also sometimes a matter of performance --- since assembly language is practically one step before machine code, program
                written in it have the minimal amount of `intermediaries', which can potentially boost it's performance.
        \end{quote}

        \subsection{Lost in translation}\label{subsection:Lost-in-translation}

        So, wouldn't it be nice if we did have some `portable' programs, which can work if not completely identically on different processors, but more or less so? Yes,
        it would be awesome. And, luckily enough, we are not the first people on Earth with such wild dreams. This is a basic idea behind a \emph{compiler} concept.

\end{document}