\documentclass[../../what-is-computer.tex]{subfiles}
\begin{document}
    \subsection{Hardware every computer needs}\label{subsection:hardware-every-computer-needs}

        This section is mostly devoted to computers as we know them today. We are surrounded by computers. Our PCs\footnote{Personal Computers}, laptops, tablets, 
        mobile phones are computers. What do they have in common? \par

        Well, first of all --- they all have different \emph{hardware}. 
        \begin{quote}
            Hardware --- is all of the computer's tangible parts or components. Pretty much everything that you can see, once you disassemble your device is hardware.
        \end{quote}

        Hardware is the basis of any computer. Once you got limited by hardware, there is not much you can do without changing that hardware. Hardware is like the lowest
        baseline measurement of what computer can do for you. To provide an analogy with a car: if your engine physically is not capable to get you 100 km/h in 4 seconds, 
        and you absolutely need to go that fast, well... change either the engine or the car itself\footnote{To change the car here means one of two thing: change
        the car entirely or change something in its configuration to get that fast, e.g. get it to be lighter}. \par

        So, \emph{hardware often acts as a physical limit of what your computer can do}. If you absolutely need to save 100 GB of photos on your computer and you just
        don't have any --- good luck. You can't just `install more disk', to get more space, you need to open up your computer case and install new disk physically. You
        can always try to somehow `shrink' photos, but that is entirely different story. \par

        \begin{quote}
            You cannot change hardware without physically installing/reinstalling some computer component. It cannot be done with some command or programm, 
            it's just screwdriver and you, buddy
        \end{quote}

        So, what hardware every computer needs? Well, maybe not every little one of them, but the majority? What will be described here sticks well with most PCs, laptops,
        phones and tablets. There are, of course, all sorts of different other computers, but, for the most cases, hardware, described here, can be adapted to them (logically)
        with very minor tweaks. \par

        \begin{enumerate}
            \item CPU\footnote{Central Processing Unit}
            \item PSU\footnote{Power Supply Unit}
            \item RAM\footnote{Rapid Access Memory}
            \item Persistent Memory storage
            \item GPU\footnote{Graphical Processing Unit}
            \item Motherboard
        \end{enumerate}
        And finally, it all should be mounted and secured in some case. Usually end-user interacts and associate hardware with case. \par

        The list of possible hardware some computer may utilize in some form is just enormous. It makes no sense to try and describe all possible parts there is.
        However, the list above is pretty comprehensive in a sense of a bare minimum for computer to function and for user to be able to interact with it. \par

        \begin{quote}
            It's worth noting, that each and every one piece of hardware described here is a complex technological device. To understand in details how works even
            one of those components --- means to research a lot of specialized literature and be capable to comprehend a lot of engineering disciplines such as
            electrical engineering and digital circuit design. This essay is intended only to introduce general functional capabilities of those devices, without
            dwelling too much into details, that a layman would find hard to understand.
        \end{quote}

    \subsection{CPU}\label{subsection:cpu}

        As was said before --- CPU is a `brain' of the computer. In the end, all operations, that somehow process any data is executed on CPU level. CPU is able to
        execute instructions, written in \emph{machine code}, presented in a \emph{binary}\footnote{Presented in $base_2$} format. Such instructions can manipulate
        memory addresses (to allocate some space in memory, or load something from it), perform arithmetic and comparative operations on 
        operands\footnote{\emph{Operand} is an object of some \emph{operation}. e.g. in `5 + 2', `+' --- is an operation, `5' and `2' --- operands}
        (usually with one or two operands), perform logic operations\footnote{Special operations such as `AND', `OR', `XOR' and others. They are subject of the discipline called `Boolean Algebra'.}, control
        flow instructions (changing order of how program is executed) and other fundamental operations. \par

        So... that's it? To learn programming --- means simply to learn those machine code instructions and find some way to shove it into CPU for it to execute?
        Well... while this would most certainly make you a programmer, there are deep caveats here.
        This question is addressed further in the chapter `\nameref{chapter:why-computer}' on~\pageref{chapter:why-computer}

    \subsection{Memory}\label{subsection:memory}
        Memory, regadless of the specific type we are talking about, serves one thing: \emph{to act as a storage}. There is, mainly, two types of memory in a 
        computer\footnote{CPU and GPU usually have their own memory. However, they are not of particular interest in this regard here}: \par
        
        \begin{enumerate}
            \item Volatile
            \item Non-volatile
        \end{enumerate}
        
        Main functional distinction between them: \emph{volatile memory} cannot save something if there is no electric power. In other words --- \emph{it empties once, the computer
        been turned off}. On the other hand --- \emph{non-volatile memory} is \emph{durable}, meaning it can store something even in case of power been cut off. \par

        Volatile memory is usually associated with RAM. Basically it's like a literal \emph{cash register}, where company stores something, to give
        out change and to which it adds today's revenue. It's accessible, fast and handy, but you wouldn't store something more substantial than one day's revenue ---
        it simply wouldn't fit. And if cashier vanishes with the register --- only today's revenue lost. \par

        Non-volatile memory is usually associated with some sort of drive. In most cases it's HDD\footnote{Hard Disk Drive}- and SSD\footnote{Solid State Drive}-like structures.
        It's \emph{the thing} that we constantly running out trying to save all the photos on our phones and laptops, or when we install a bunch of programs on our computers.
        To continue our association --- it's like a \emph{safe deposit box}, or a \emph{bank vault}. It's secure and suitable for long-term storage. Disaster can still happen,
        of course, but it's magnitude must be considerate for the bank to fail us, opposite to cash register scenario. However it's a bit of a hustle to go and put something 
        in a said \emph{vault}. You wouldn't go to it every 5 minutes to give out change from 5\$, would you? No, you would rather \emph{accumulate} substantial amount
        in your cash register first, then go to the bank and deposit it. \par

        This simple analogy sums up why we have 2 types of memory: volatile is fast, but small and not much stable and non-volatile is stable and large, but slow. We combine them to achieve what we
        consider optimal performance.

    \subsection{GPU}\label{subsection:gpu}
    \subsection{Motherboard and PSU}\label{subsection:motherboard-and-psu}
\end{document}