\documentclass[../../what-is-computer.tex]{subfiles}
\begin{document}
    \subsection{Hardware every computer needs}\label{subsection:hardware-every-computer-needs}

    This section is mostly devoted to computers as we know them today. We are surrounded by computers. Our PCs\footnote{Personal Computers}, laptops, tablets, 
    mobile phones are computers. What do they have in common? \par

    Well, first of all --- they all have different \emph{hardware}. 
    \begin{quote}
        Hardware --- is all of the computer's tangible parts or components. Pretty much everything that you can see, once you disassemble your device is hardware.
    \end{quote}

    Hardware is the basis of any computer. Once you got limited by hardware, there is not much you can do without changing that hardware. Hardware is like the lowest
    baseline measurement of what computer can do for you. To provide an analogy with a car: if your engine physically is not capable to get you 100 km/h in 4 seconds, 
    and you absolutely need to go that fast, well... change either the engine or the car itself\footnote{To change the car here means one of two thing: change
    the car entirely or change something in its configuration to get that fast, e.g. get it to be lighter}. \par

    So, \emph{hardware often acts as a physical limit of what your computer can do}. If you absolutely need to save 100 GB of photos on your computer and you just
    don't have any --- good luck. You can't just `install more disk', to get more space, you need to open up your computer case and install new disk physically. You
    can always try to somehow `shrink' photos, but that is entirely different story. \par

    \begin{quote}
        You cannot change hardware without physically installing/reinstalling some computer component. It cannot be done with some command or programm, 
        it's just screwdriver and you, buddy
    \end{quote}

    So, what hardware every computer needs? Well, maybe not every little one of them, but the majority? What will be described here sticks well with most PCs, laptops,
    phones and tablets. There are, of course, all sorts of different other computers, but, for the most cases, hardware, described here, can be adapted to them (logically)
    with very minor tweaks. \par

    \begin{enumerate}
        \item CPU\footnote{Central Processing Unit}
        \item PSU\footnote{Power Supply Unit}
        \item RAM\footnote{Rapid Access Memory}
        \item Persistent Memory storage
        \item GPU\footnote{Graphical Processing Unit}
        \item Motherboard
    \end{enumerate}
    And finally, it all should be mounted and secured in some case. Usually end-user interacts and associate hardware with case. \par

    The list of possible hardware some computer may utilize in some form is just enormous. It makes no sense to try and describe all possible parts there is.
    However, the list above is pretty comprehensive in a sense of a bare minimum for computer to function and for user to be able to interact with it. \par

    \begin{quote}
        It's worth noting, that each and every one piece of hardware described here is a complex technological device. To understand in details how works even
        one of those components --- means to research a lot of specialized literature and be capable to comprehend a lot of engineering disciplines such 
        electrical engineering and digital circuit design. This essay is intended only to introduce general functional capabilities of those devices, without
        dwelling too much into details, that a layman would find hard to understand.
    \end{quote}

    \subsection{CPU}\label{subsection:cpu}

    As was said before --- CPU is a `brain' of the computer. In the end, all operations, that somehow process any data is executed on CPU level. CPU is able to
    execute instructions, written in \emph{machine code}, presented in a \emph{binary}\footnote{Presented in $base_2$} format. Such instructions can manipulate
    memory addresses (to allocate some space in memory, or load something from it), perform arithmetic and comparative operations on 
    operands\footnote{\emph{Operand} is an object of some \emph{operation}. e.g. in `5 + 2', `+' --- is an operation, `5' and `2' --- operands}(usually with one or two operands).
    perform logic operations\footnote{Special operations such as `AND', `OR', `XOR' and others. They are subject of the discipline called `Boolean Algebra'.}, control
    flow instructions (changing order of how program is executed) and other fundamental operations. \par

    So... that's it? To learn programming --- means simply to learn those machine code instructions and find some way to shove it into CPU for it to execute?
    Well... yes and no. This question is addressed further in the chapter `\nameref{chapter:why-computer}' on~\pageref{chapter:why-computer}

    \subsection{Memory}\label{subsection:memory}
    \subsection{GPU}\label{subsection:gpu}
    \subsection{Motherboard and PSU}\label{subsection:motherboard-and-psu}
\end{document}