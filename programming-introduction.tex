\documentclass{article}

\usepackage{hyperref}
\hypersetup{colorlinks,%
            citecolor=black,%
            filecolor=black,%
            linkcolor=black,%
            urlcolor=black,%
            pdftex}

%Title section
\title{What is programming?}
\author{Khambar Dussaliyev}
\date{\today}

\begin{document}
    \maketitle
    \begin{abstract}
        This article is intended to answer (very briefly) to a question: \emph{What is a programming?}
        Although computer programs nowadays are pretty ubiquitous in nature, they remain being \emph{black boxes of magic}
        for most people, even tech-savvy ones.
        \\
        However, being a \emph{black box of magic} to some extent is a requirement for some commercial software (especially when there are trade secrets involved) 
        natural. But this article is not intended to answer how every software works in details, but \emph{how every software works, in general}.
    \end{abstract}
    \newpage

    \tableofcontents
    \newpage

    \section{Introduction}
        Basically, \emph{everything} you do on a computer is running some program. Everything, no matter what you do is a result of executing and working with
        some program. Your OS\footnote{Operation System - Windows, Linux-based, MacOS, Android, iOS etc.} \emph{is a program}. Your internet browser is a program. 
        Your media player, file explorer, video game, media editing software, office software \emph{are programs}. 
        \emph{Everything that allows you to interact\footnote{Interaction with a computer here and on will not take physical interaction with a bare metal in account} 
        with computer (and not just bare metal) is a program.}

        So, essentially, \emph{what is a program?}. According to Marriam-Webster definition\footnote{\href{https://www.merriam-webster.com/dictionary/program}{https://www.merriam-webster.com/dictionary/program}} 
        (applicable for us), we can use following as a definition:
        
        \begin{quote}
            program is a sequence of coded instructions that can be inserted into a mechanism (such as a computer)
        \end{quote}

        The gist of it being -- \emph{sequense of instructions}. So every interaction we can possibly have with a computer is somehow just a set of instructions. But \emph{how computers
        do understand out instructions}? If I just shout into the microphone some command, computer will not just do as I say\footnote{Provided, there is no running program, responsible for such behavior}.
        The same effect will have some instruction that I carefully write them in some text document, using Microsoft Word, for example.
        \\ 
        How do I make computer to understand what I want from it? To understand this, \emph{we must first understand what is a computer}.

    \section{What is Computer?}


    
\end{document}