\documentclass{article}

%Hyperref must be the last dependency
\usepackage{hyperref}
\hypersetup{colorlinks,%
            citecolor=black,%
            filecolor=black,%
            linkcolor=black,%
            urlcolor=black,%
            pdftex
}


%Title section
\title{What is programming?}
\author{Khambar Dussaliyev}
\date{\today}

%Document
\begin{document}
    \maketitle
    \begin{abstract}
        This essay is intended to answer (very briefly) to a question: \emph{What is a programming?}
        Although computer programs nowadays are pretty ubiquitous in nature, they remain being \emph{black boxes of magic}
        for most people, even tech-savvy ones.  
        \\
        However, being a \emph{black box of magic} to some extent is a requirement for some commercial software (especially when there are trade secrets involved) 
        natural. But this essay is not intended to answer how every software works in details, but \emph{how every software works, in general}.
    \end{abstract}
    \newpage

    \tableofcontents
    \newpage

    \section{Introduction}

        Basically, \emph{everything} you do on a computer is running some program. Everything, no matter what you do is a result of executing and working with
        some program. Your OS\footnote{Operation System - Windows, Linux-based, MacOS, Android, iOS etc.} \emph{is a program}. Your internet browser is a program. 
        Your media player, file explorer, video game, media editing software, office software \emph{are programs}. 
        \emph{Everything that allows you to interact\footnote{Interaction with a computer here and on will not take physical interaction with a bare metal in account} 
        with computer (and not just bare metal) is a program.}
        
        So, essentially, \emph{what is a program?}. According to Marriam-Webster\footnote{Here and on Marriam-Webster dictionary is referred to as a `general-scope' dictionary to avoid technical details redundant for this essay} 
        definition\footnote{\href{https://www.merriam-webster.com/dictionary/program}{https://www.merriam-webster.com/dictionary/program}} 
        (applicable for us), we can use following as a definition:
        
        \begin{quote}
            program is a sequence of coded instructions that can be inserted into a mechanism (such as a computer)
        \end{quote}

        The gist of it being -- \emph{sequense of instructions}. So every interaction we can possibly have with a computer is somehow just a set of instructions. But \emph{how computers
        do understand out instructions}? If I just shout into the microphone some command, computer will not just do as I say\footnote{Provided, there is no running program, responsible for such behavior}.
        The same effect will have some instruction that I carefully write them in some text document, using Microsoft Word, for example.
        \\ 
        How do I make computer to understand what I want from it? To understand this, \emph{we must first understand what is a computer}.

    \section{What is Computer?}

        Let's once again refer to Marriam-Webster dictionary for a \emph{computer} definition\footnote{\href{https://www.merriam-webster.com/dictionary/computer}{https://www.merriam-webster.com/dictionary/computer}}:

        \begin{quote}
            computer is a programmable usually electronic device that can store, retrieve, and process data.
        \end{quote}

        So, a computer directly tied to all sorts of \emph{data maniplation}. But what is data, and how can it be manipulated? Data is pretty much any factual information.
        Weather outside a window? Data. T-Shirt you're wearing? Data. Dusty books in my shelves? Data. But there are a certain layers present. The fact that I'm wearing a T-Shirt 
        is data (even if I'm not -- also data). What color it is is also, most certainly data. Is there any text or image present?
        Both the existence and \emph{information} in it -- also data. Let's also not forget about colors, sizes, fonts. We are living and have always lived in an enormous ocean of data. Nowadays, with our technology even more so.
        \\
        But do we have use for this data? Well, the answer is -- it depends. To assess data without well-defined goal will almost always result in you drowning in said data without much progress,
        since world offers us practically indefinite source of data. We must put our data into some perspective, some context. Once we put our data in some context and it becomes \emph{useful} for us, \emph{it becomes information}.
        
        \begin{quote}
            To somehow navigate in this world, we must put our data in context. Data with given context becomes somewhat useful. Such data called \emph{information}.
        \end{quote}

        %Pascal and Pascaline
        %Charles Babbidge
        %Ada Lovelace https://www.youtube.com/watch?v=IZptxisyVqQ
        %Jacquard Loom and punchcards https://www.youtube.com/watch?v=K6NgMNvK52A
        %Programming languages
        

    
\end{document}

