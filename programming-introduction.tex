\documentclass{article}

%Hyperref must be the last dependency
\usepackage{hyperref}
\hypersetup{colorlinks,%
            citecolor=black,%
            filecolor=black,%
            linkcolor=black,%
            urlcolor=black,%
            pdftex
}


%Title section
\title{What is programming?}
\author{Khambar Dussaliyev}
\date{\today}

%Document
\begin{document}
    \maketitle
    \begin{abstract}
        This essay is intended to answer (very briefly) to a question: \emph{What is programming?}
        Although computer programs nowadays are pretty ubiquitous in nature, they remain being \emph{black boxes of magic}
        for most people, even tech-savvy ones. \par
        
        However, being a \emph{black box of magic} to some extent is a requirement for some commercial software (especially when there are trade secrets involved) 
        natural. But this essay is not intended to answer how every software works in details, but \emph{how every software works, in general}.
    \end{abstract}
    \newpage

    \tableofcontents
    \newpage

    \section{Introduction}

        Basically, \emph{everything} you do on a computer is running some program. Everything, no matter what you do is a result of executing and working with
        some program. Your OS\footnote{Operation System - Windows, Linux-based, MacOS, Android, iOS etc.} \emph{is a program}. Your internet browser is a program. 
        Your media player, file explorer, video game, media editing software, office software \emph{are programs}. 
        \emph{Everything that allows you to interact\footnote{Interaction with a computer here and on will not take physical interaction with a bare metal in account} 
        with computer (and not just bare metal) is a program.}
        
        So, essentially, \emph{what is a program?}. According to Marriam-Webster\footnote{Here and on Marriam-Webster dictionary is referred to as a `general-scope' dictionary to avoid technical details redundant for this essay} 
        definition\footnote{\href{https://www.merriam-webster.com/dictionary/program}{https://www.merriam-webster.com/dictionary/program}} 
        (applicable for us), we can use following as a definition:
        
        \begin{quote}
            program is a sequence of coded instructions that can be inserted into a mechanism (such as a computer)
        \end{quote}

        The gist of it being -- \emph{sequense of instructions}. So every interaction we can possibly have with a computer is somehow just a set of instructions. But \emph{how computers
        do understand out instructions}? If I just shout into the microphone some command, computer will not just do as I say\footnote{Provided, there is no running program, responsible for such behavior}.
        The same effect will have some instruction that I carefully write them in some text document, using Microsoft Word, for example.
        \\ 
        How do I make computer to understand what I want from it? To understand this, \emph{we must first understand what is a computer}.
        \newpage
    \section{What is Computer?}

        \subsection{It's all about information}
            Let's once again refer to Marriam-Webster dictionary for a \emph{computer} definition\footnote{\href{https://www.merriam-webster.com/dictionary/computer}{https://www.merriam-webster.com/dictionary/computer}}:

            \begin{quote}
                computer is a programmable usually electronic device that can store, retrieve, and process data.
            \end{quote}

            So, a computer directly tied to all sorts of \emph{data maniplation}. But what is data, and how can it be manipulated? Data is pretty much any factual information.
            Weather outside a window? Data. T-Shirt you're wearing? Data. Dusty books in my shelves? Data. But there are a certain layers present. The fact that I'm wearing a T-Shirt 
            is data (even if I'm not -- also data). What color it is is also, most certainly data. Is there any text or image present?
            Both the existence and \emph{information} in it -- also data. Let's also not forget about colors, sizes, fonts. We are living and have always lived in an enormous ocean of data. Nowadays, with our technology even more so. \par
            
            But do we have use for this data? Well, the answer is -- it depends. To assess data without well-defined goal will almost always result in you drowning in said data without much progress,
            since world offers us practically indefinite source of it. We must put our data into some perspective, some context. Once we put our data in some context and it becomes \emph{useful} for us, \emph{it becomes information}.
            
            \begin{quote}
                To somehow navigate in this world, we must put our data in context. Data with given context becomes somewhat useful. Such data called \emph{information}.
            \end{quote}

            Information is much more well defined than data. \emph{We can measure information, actually}. We even have a science discipline, called \emph{Information theory},
            that researches all about information, from mathematical and engineering standpoint. Not only that, but we have an entire industry, called \emph{Information Technologies}
            built entirely on a foundation provided by Information theory and adjacent disciplines.\par
            
            But until we dwell into technical stuff, we must also remember, that \emph{we used to operate with information}. Our brain is a natural computer, disecting data
            into information that we use in our everyday life. How come I am so sure to call that information and not just raw data? Well, that greatly depends on a 
            scale, but
            \emph{our brain have very defined goals}. One of the main said goals being to \emph{keep us alive}. Therefore there is a context, and brain will always tend to 
            categorize things (organize data) based on this goal\footnote{If your goal at the moment being something more specific, than just stay alive, many of the information brain gives you still raw data},
            making it somewhat useful, therefore making it an \emph{information}. Even the simple fact, that we don't notice our noses, although our eyes do see it constantly, 
            tells us how natural brain is in working with information. \par
            
            So, being a natural organic computer, we must first understand what we do with information, to have insight on what computers do with information. All processing 
            of information, our minds in work is mostly an internal process. \emph{Sooner or later there is a point, when we must exchange said information}. So, how do we
            exhange it? We have an enormous number of ways, actually. We can say something to another person, write it down, pass via somebody a note, we can just hint at something.
            We can send message in a messenger, send an email, send a radio-signal, we can knock a Morse code. \emph{We are practically limitless with one major nuance -- the other
            side must be able to understand us}. There is no point in sending email to a person, who can't use it\footnote{In general. Sometimes we are interested only
            in sending information, not concerned by an actual delivery. Some legal procedures can be of an example}. \par

            We now can conclude one fundamental distinction: information itself is mostly independent from it's carrier. To demonstrate it more clearly, let us consider
            following example: I want to send information to my friend at the table, with the main message being \emph{`pass me salt'}. There is definetely a context:
            we are at the table, eating food, there are at least two parties involved (me and my friend), and I expect salt to exist somewhere at the table. I can pass this 
            information with a various number of ways, provided my friend understands me. Just to mention a few:

            \begin{itemize}
                \item Say to him `pass me salt'
                \item Ask him to pass me salt in other language, he is familiar with
                \item Write a note to him `pass me salt'
                \item Write a note in foreign language, he is familiar with
                \item Get his attention and non-verbally point at salt
                \item Write him a message in messenger, expecting phone to be near them
                \item Get his attention and use ASL or alternative, provided he is familiar with it
                \item Exclaim obviously `Oh! This food will be so much better with salt! I wish somebody passed it to me now', provided he understood our hint
                \item Rhytmically knock with Morse code, provided he understands it
            \end{itemize}

            In all aforementioned examples we can clearly see, that the gist of our `message' stayed the same. \emph{We did pass a more or less the same information} in 
            each and every case. Despite the medium being completely different, if our friend can undestand us, nothing really changed for him or us. In such cases
            the `main message' containing an actual useful information we are willing to exchange usually colloquially called a \emph{`payload'}. However,
            despite our `payload' being virtually the same, we did pass some additional information (or data -- depending on context) along the way, didn't we?

            Let's use an above list one more time, but will provide additional few details, just for example:

            \begin{itemize}
                \item Say to him `pass me salt'
                \begin{itemize}
                    \item In what voice tone?
                    \item How loud did we ask?
                    \item What face expressions followed along our request?
                    \item Was there any gesticulation involved? How intense?
                    \item In what speed we asked?
                \end{itemize}

                \item Ask him to pass me salt in other language, he is familiar with
                \begin{itemize}
                    \item What language?
                    \item Was there any context in using this language?
                    \item In what voice tone?
                    \item How loud did we ask?
                    \item What face expressions followed along our request?
                    \item Was there any gesticulation involved? How intense?
                    \item In what speed we asked?
                \end{itemize}

                \item Write a note to him `pass me salt'
                \begin{itemize}
                    \item What font did we use?
                    \item Is it hand-written?
                    \item Font color?
                    \item Font size?
                    \item Paper type?
                    \item Paper size?
                    \item Was paper plain white, or was it with pictures?
                \end{itemize}

                \item Write a note in foreign language, he is familiar with
                \begin{itemize}
                    \item What language?
                    \item Was there any context in using this language?
                    \item What font did we use?
                    \item Is it hand-written?
                    \item Font color?
                    \item Font size?
                    \item Paper type?
                    \item Paper size?
                    \item Was paper plain white, or was it with pictures?
                \end{itemize}

                \item Get his attention and non-verbally point at salt
                \begin{itemize}
                    \item Were we mumbling at the same time?
                    \item How we got his attention? Tapped his shoulder? How strongly?
                    \item How did we point? With a finger, palm, node?
                    \item What face expressions have we used?
                    \item How fast did we do it? 
                \end{itemize}

                \item Write him a message in messenger, expecting phone to be near them
                \begin{itemize}
                    \item Have he seen us typing a message?
                    \item How fast he reacted?
                    \item Was there any notification
                    \item Did we send an emoji?
                    \item Did we send some attachment?
                    \item We sent one message or several?
                    \item Did he read it?
                \end{itemize}

                \item Get his attention and use ASL or alternative, provided he is familiar with it
                \begin{itemize}
                    \item How exactly did we phrase our request? By letters or by gests?
                    \item How fast did we transmit?
                    \item We followed along with our lips?
                \end{itemize}

                \item Exclaim obviously `Oh! This food will be so much better with salt! I wish somebody passed it to me now', provided he understood our hint
                \begin{itemize}
                    \item What intonation did we use?
                    \item How loud did we say it?
                    \item Where to did we look?
                    \item Do we have any specific accents?
                    \item Was there an emphasis on some words?
                \end{itemize}

                \item Rhytmically knock with Morse code, provided he understands it
                \begin{itemize}
                    \item What period of time we used as an interval?
                    \item Did we repeat our message? How many times?
                    \item On what surface did we transmit?
                    \item Did we use our knuckle? Spoon? Knife?
                \end{itemize}
            \end{itemize}
            
            So, we can conclude, that despite our \emph{payload} being practically the same, we did pass additional information along with it. To put it into perspective,
            It's somewhat similar, as if we were asked to describe an envelope, it's size, stamps on it and additional notes, ignoring the payload, being a letter inside
            said envelope. Such auxiliary information, which is more often than not isn't of our interest, however \emph{can be useful in certain scenarios}. Such data
            usually describe optional information aboout the \emph{payload} itself, or details of how it was delivered. Such data usually called \emph{metadata}

            \begin{quote}
                The information itself, that we wish to store or exchange colloquially called a \emph{payload}. Some additional details, that might be useful, regarding
                that information, but not tied to it directly usually called \emph{metadata}
            \end{quote}

            Let's say, I am writing a document in Microsoft Word. The \emph{payload} here being anything, I typed directly in this document. However, once I saved it, not
            only the document itself was saved, but also a bunch of additional \emph{metadata}. It can include date of the document creation, 
            author\footnote{Usually currently active user on OS level} of the document, last save date, last print date, etc. \emph{The same logic is applicable for virtually
            any file you've ever created}.
            \footnote{Sometimes, ability to control metadata becomes crucial to save sensitive and personal information. 
            Some professions can put you in physical danger, if you're not cautious enough} \par

            \newpage
        \subsection{You can't manage what you can't measure}

            \newpage
        \subsection{Countess, weaving patterns and count machines}

            \newpage
        \subsection{But can it speak?}
            \newpage


        

        %Pascal and Pascaline
        %Charles Babbidge
        %Ada Lovelace https://www.youtube.com/watch?v=IZptxisyVqQ
        %Jacquard Loom and punchcards https://www.youtube.com/watch?v=K6NgMNvK52A
        %Programming languages
        

    
\end{document}

